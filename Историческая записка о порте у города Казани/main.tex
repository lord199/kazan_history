
\documentclass[oneside,final,14pt]{extreport}
%\usepackage[koi8-r]{inputenc}
\usepackage[russianb]{babel}
\usepackage{vmargin}
\setpapersize{A4}
\usepackage[T2A]{fontenc}
\usepackage[utf8x]{inputenc}  % more recent versions (at least>=2004-17-10)
%\usepackage[russian]{babel}
\setmarginsrb{2cm}{1.5cm}{1cm}{1.5cm}{0pt}{0mm}{0pt}{13mm}
\usepackage{indentfirst}
\sloppy
\begin{document}


\section*{Историческая записка о порте у города Казани}


\begin{center}
	I
\end{center}

Вопрос о сооружении бухты в Казани был возбужден еще задолго до введения Городского Положения 16 июня 1870 года (по словам Н. Ф. Банарцева в 1858 году), но осуществление его тестно связывалось с проведением на Казань железнодорожной линии, почему и откладывалось из года в год. Тем не менее в 1872 году частной компании Арцыбушева предоставлена была возможность взяться за это предприятие. Однако предприятие это не удалось по причинам не зависившим от Городского Управления, так как на это дело потребовались средства, оказавшиеся не по силам названной компании.

После этой неудачи, постигшей частную предприимчивость, в 1886 году Городскою Думой возбуждено было ходатайство пред Правительством об устройстве бухты на средства Государственного Казначейства, но ходатайство это решено было в отрицательном смысле.


\end{document}
