
\documentclass[oneside,final,14pt]{extreport}
%\usepackage[koi8-r]{inputenc}
\usepackage[russianb]{babel}
\usepackage{vmargin}
\setpapersize{A4}
\usepackage[T2A]{fontenc}
\usepackage[utf8x]{inputenc}  % more recent versions (at least>=2004-17-10)
%\usepackage[russian]{babel}
\setmarginsrb{2cm}{1.5cm}{1cm}{1.5cm}{0pt}{0mm}{0pt}{13mm}
\usepackage{indentfirst}
\usepackage{nicefrac} % For comparison
%\usepackage{xfrac}    % Works better with other fonts
\usepackage[unicode, pdftex]{hyperref}
\sloppy
\begin{document}
	
	\section*{Описание}
	
	{\bf Название:} Историческая записка о порте у города Казани
	
	{\bf Издатель:} Казань Казанский федеральный университет 2013
	
	{\bf Предметные рубрики:} Казанский речной порт -- История
	
	{\bf Инвентарный номер}: RU05CLSL05CORRK05C1925 / c8470
	
	{\bf Ссылка:} \hyperref[http://repo.kpfu.ru/jspui/handle/net/4754]{http://repo.kpfu.ru/jspui/handle/net/4754}
	
	{%
		\centering
		\section*{Историческая записка \\о порте у города Казани}
	}
	
	\begin{center}
		\noindent\rule{8cm}{0.4pt}
	\end{center}
	
	{%
		\centering
		\subsection*{I.}
	}
	
	Вопрос о сооружении бухты в Казани был возбужден еще задолго до введения Городского Положения 16 июня 1870 года (по словам Н. Ф. Банарцева в 1858 году), но осуществление его тестно связывалось с проведением на Казань железнодорожной линии, почему и откладывалось из года в год. Тем не менее в 1872 году частной компании Арцыбушева предоставлена была возможность взяться за это предприятие. Однако предприятие это не удалось по причинам не зависящим от Городского Управления, так как на это дело потребовались средства, оказавшиеся не по силам названной компании.
	
	После этой неудачи, постигшей частную предприимчивость, в 1886 году Городскою Думой возбуждено было ходатайство пред Правительством об устройстве бухты на средства Государственного Казначейства, но ходатайство это решено было в отрицательном смысле.
	
	Хотя Дума снова поддерживала сношение с частными предпринимателями по сооружению бухты на концессионных началах, но все многочисленные попытки и старания к подысканию частных капиталистов не увенчались успехом.
	
	В 1886 году вопрос о выгодности бухты для Казани в финансовом и техническом отношениях был тщательно рассмотрен и обсужден в особой комиссии при местном Округе Путей Сообщения под председательством Начальника Округа Августовского при участии представителей: Казанской Думы, Биржевого Комитета, пароходчиков и купечества, причем Комиссия эта пришла к единогласному решению, что бухта для Казани безусловно необходима, что она даст городу пользы до 150000 руб. в год, что в техническом отношении сооружение бухты не встречает препятствий и, что, наконец,  полная стоимость этого сооружения эллингом определится в сумму около 180000 руб. Подобная записка этой Комиссии препровождена в Министерство Путей Сообщени, где она находится по сие время (в 1889 г. в бытность Городского Головы в Петербурге, он справлялся о судьбе этой записки - оказалось, что она еще не рассматривалась в Министерстве). 
	
	В 1891 году было предложение некоего Миллиса о принятии этого сооружения к исполнению на средства предполагавшегося к учреждению акционерного общества при условии гарантии со стороны города ежегодного валового дохода в 100000 руб. Дума, рассмотрев это предложение, признала невыгодным его для города и отклонила.
	
	Несмотря на полное сознание всего населения г. Казани как необходимости, так и полезности сооружения бухты, вопрос о ней оставался неподвижным.
	
	Более реальную фазу вопрос о сооружении бухты принял в 1893 году, благодаря почину гласного Думы В. М. Ключникова. В обстоятельной записке г. Ключников, путем изучения всех имеющихся по этому вопросу данных, доказал все преимущества этого предприятия для гор. Казани, как по топографическим её условиям, так и по выгодности сооружения бухты средствами города, причем он предложил ходатайствовать, чтобы бухта была построена Министерством Путей Сообщения, но на средства города. Записка эта обсуждалась состоявшею при Городской Управе Техническою Комиссией, при участии и содействии в разработке детальных сторон этого вопроса представителем Казанского Округа Путей Сообщения, во главе с Помощником Начальника Округа, инженером Макаровым. Подробно обсудив способы сооружения бухты, Комиссия приняла следующие основные положения: 
	
	\begin{enumerate} 
		\item Канал должен быть шириною в 35 саж., глубиною в 12 четвертей от низкого горизонта воды (при 0,30 ниже нуля Услонской рейки);
		\item Сама бухта должна быть удалена от границы частных владений города на 200 саж.;
		\item Ширину проездов в бухте определить во всех пунктах не менее 100 саж. низшего горизонта вод, при глубине в 12 четвертей, указанных в отношении канала в меженную воду;
		\item Длина береговой линии кругом бухты должна быть около 1500 саж.;
		\item На острове, среди бухты, должно быть достаточно места для служебных зданий, эллингов и доков;
		\item Ширина насыпи в боковых и нижних частях бухты должна быть не менее в 100 саж. и со стороны города до 200 саж.;
		\item Кругом набережная должна быть устроена по типу Нижегородской набережной с двойным объездом, при этом, однако, высоту нижней бермы следует взять соответственно естественному возвышению местности с тем, чтобы впоследствии, в случае надобности, она могла быть понижена. С нижней бермы должно быть сделано по крайней мере два пологих двойных въезда на верхнюю набережную. Ширина бермы должна быть в 10 саж.;
		\item  Ширина мостовых определяет так: а) весь нижний откос от воды, б) 6 сажень нижней бермы под дорогу и в) верхний путь не менее 10 сажень;
		\item Признавая всю важность и полезность эллингов и доков, в виду их дороговизны, предоставить устройство их, одновременно с сооружением бухты, частной предприимчивости: вызвав на эти сооружения особых предпринимателей;
		\item Высота всех насыпей, кругом бухты должна быть выше высокого уровня воды, бывшего в 1867 году, на 6 четвертей;
		\item От города должны быть сделаны следующие сооружения: дом для администрации, будки, фонари, пожарное депо с каланчей и другие необходимые сооружения.
	\end{enumerate}
	
	Обращаясь к материальной стороне вопроса, Комиссия пришла к решительному и единогласному заключению, что единственный, более рациональный путь в данном случае для интересов и пользы дела {\it строить бухту на счет города, средствами займа}, причем путем подробных исчислений особой подкомиссии; доходность бухты установлена следующими цифрами: с земель набережной 90000 руб., попудный сбор 50000 руб. и за зимовку судов 8000 руб., а всего 148000 руб. Расходы же по эксплуатации бухты ожидаются по вычислению подкомиссии следующие: а) процентов на капитал (принимая стоимость сооружения в 100000 руб.) с погашением около 50000 руб., собственно расходов на эксплуатации вместе с ремонтов 35000 руб., всего 85000 руб. Таким образом, если исключить из ожидаемой прибыли от бухты доходы, получаемые с пристаней (в 1893 г., во время составления настоящего расчета), даже при крайне осторожном исчислении, как доходов, так равно и расходов, — получится полнейшая выгода для города от означенного предприятия.
	
	Руководствуясь принятыми выше соображениями, Комиссия, в виду крайне сочувственного отношения местного Округа Путей Сообщения к делу осуществления означенного предприятия, обратилась к г. Начальнику Округа с просьбою поручить инженерам вверенного ему Округа составить проект сооружения бухты на основании данных, выработанных Комиссией с помощью принимавших участие в её трудах специалистов.
	
	Начальник Округа Лохтин, препровождая разработанный проект бухты, составленный инженерами Округа, в письме на имя бывшего в то время Городским Главой С. В. Дьяченко, выразил удовольствие, что в данном случае ему удалось быть полезным в деле осуществления этого предприятия, которое он надеется "при нынешнем" сочувствии этому делу со стороны Городской Думы и сознания всего его значения как для города, так и для судоходства, не встретив в дальнейшем каких-либо препятствий или отсрочек. Не подлежит сомнению, писал г. Лохтин, что порт этот стоящий на рубеже двух волжских плес, верхнего и нижнего, естественным образом различающихся своим характером и глубиною и примыкающий с одной стороны к вновь открытой железной дороге, которая соединяет его непосредственно с Москвою и с другой - с восточными рынками Камского бассейна, послужив весьма важным сооружением для всего волжского судоходного движения и, в то же время, щедро вознаградить за заботы о своем осуществлении и сам город, оживлением его торговли, открытием механических мастерских и заводов для ремонта или постройки судов, привлечением к порту многочисленного нового народонаселения и т.п. Не даром вопрос о порте в Казани, возникший почти полстолетия тому назад, благодаря своей жизненности, не переставал время от времени обращать на себя внимание и создал за этот период времени целый ряд проектов и предложения по поводу осуществления этого дела. Остается выразить только удивление, дальше пишет г. Лохтин, что столь серьезный и так сказать назревший городской вопрос оставался до сего времени не разрешенным, несмотря на очевидную необходимость в этом сооружении с точки зрения городских интересов. Препровождаемый проект, составлен на основании тщательных и подробных исследований местности и грунтов и разработан при участии большинства Окружных Инженеров, обсуждавших его в общих своих совещаниях, а потому г. Лохтин рекомендует этот проект, как достаточно гарантирующий правильность его разработки с технической точки зрения. Общая стоимость проекта высчитана в сумме 1157142 руб.
	
	Казанская Городская Дума, в заседании 7 июня 1894 г., после всестороннего рассмотрения означенного вопроса, вполне согласившись с вышеприведенными доводами Технической Комиссии, признала, что если стоимость сооружения будет в действительности несколько дороже против проекта, выработанного Округом Путей Сообщения и бухта будет стоить городу даже полтора миллиона рублей, то при займе этой суммы из 6 процентов с погашением капитального долга, ежегодный платеж города будет равен 90000 руб. Следовательно, покрыв этот годичный платеж долга выручкой от береговых участков, город будет иметь чистым доходом сборы: за зимовку, за плавание, за нагрузку и выгрузку товаров и проч., а через несколько десяткой лет -- городу останется бухта совершенно свободная от долгов. На основании сих соображений, Дума постановила: 1) уполномочить Городскую Управу возбудить ходатайство перед Правительством о разрешении Казанскому Городскому Общественному Управлению долгосрочного займа в сумму 1500000 руб. из 5\nicefrac{1}{2} или 6\% годовых с погашением капитала в течение от 36 до 50 лет на предмет сооружения волжской бухты; 2) поручить Управе озаботиться приисканием капиталистов, которые согласились бы дать городу потребную вышеуказанную сумму в случае разрешения займа Правительством и 3) признать, что за Городскою Думой остается право воспользоваться займом в случае его разрешения в полной сумме, или в части, или вовсе отказаться от займа.
	
	Изложенное в постановлении ходатайство Думы для дальнейшего направления было представлено г. Губернатору 1 августа 1894 г. за № 7906.
	
	Составленный Инженером Округа Путей Сообщения проект волжской бухты, вследствие требования г. Казанского Губернатора, был предоставлен Его Превосходительству при предоставлении от 17 сентября 1094 г. за № 9566.
	\begin{center}
		\noindent\rule{4cm}{0.4pt}
	\end{center}
	
	Затем, Городская дума, заслушав в заседании 27 сентября 1900 г. заявление французского гражданина Жоржа Мусси по вопросу о сооружении в г. Казани бухты, постановила: уведомить г. Мусси, что сооружение волжской бухты Городская Дума считает делом первостепенной важности и предоставляет ему за свой счет и страх, без всякого участия со стороны города в расходах, производить изыскания и подготовительные работы, давая вместе с тем обещание уведомить его обо всяком могущем последовать в течение года от сего числа предложения со стороны кого-либо из предпринимателей по делу об устройстве бухты.
	
	Изложенное постановление Городской Думы сообщено г. Мусси 9 октября 1900 г. за № 11025.
	
	{%
		\centering
		\subsection*{II.}
	}
	При вступлении на должность нового Городского Головы С. А. Бекетова, в первом коллегиальном совещании Член Городской Управы, заведующий Волжскими пристанями, вновь возбуждает вопрос об необходимости порта у города Казани, проводя в своем докладе все неотложные мотивы сооружения, — железный путь на Уфу нанес самый тяжкий удар нашему благосостоянию, со дня открытия этой дороги, сразу понизились все доходы Казани, более, чем на половину. Причина естественного вымирания такого большого культурного города, как Казань, для нас очевидна, общегосударственный экономический кризис ударил самым больным своим концом по нашему городу. Казань была сильна по счастливому своему географическому положению, занимая центральный пункт на Волге. Казань издревле была теми воротами, через которые обширные и богатый Восток России обменивался со всею Западною Европою. К Казани тянула вся Сибирь свой великой транзитной артерией, для Сибири Казань играла ту же роль, как Москва для Европейской России. Первенствующую роль, Казань утратила навсегда по случаю перемещения торговых путей и рынков, мы лишились помимо нашей воли груза, а теперь отсутствие грузоспособности представляется стимулом, доказывающий невыгодность проведенных рельсовых путей. Пора уже нам действовать собственным трудом, собственною предприимчивостью, неуклонно стремится к поддержанию первенствующего значения Казани на водном пути, всецело воспользоваться своим выгодным естественным положением, которое дано нам природою среди протекающих в нашем крае судоходных рек.
	
	Очевидно, что такое важное экономическое положение Казань может только иметь при помощи сооружения в нашем городе порта.
	
	Сооружение порта нужно считать {\it краеугольным} камнем в развитии Казанской промышленности. Порт, вместе с мостом через р. Волгу, создадут громадное перерождение нашему оскудевшему город; промышленность его устремится к повышению, народонаселение год от года будет страшно увеличиваться, - одним словом, Казань на Волге благодаря своей культурности, прославится своими торговыми оборотами на всю Россию. Благодаря будущим путям железнодорожных линий, Казанский Порт превратится в место скрещения товаров от которого последние будут расходиться по различным направлениям и, благодаря выгодности расположения порта, между двух богатых бассейнов Волги и Камы, он будет иметь возможность удовлетворять потребностям судоходства и нуждам торговых сношений.
	
	12 ноября прошлого года Городской Голова и Члены Городской Управы возлагают на г. Банарцева трудную, но вместе с тем очень важную для города задачу по составлению предварительного проекта порта.
	
	\newpage
	
	{%
		\centering
		\section*{Пояснительная записка \\сооружения порта у г. Казани\\по проекту Н. Ф. Банарцева.}
	}
	
	Самым существенным недостатком Казанских пристаней следует признать недостаточную высоту Волжского берега у Казани, благодаря чему он каждую весну бывает покрыт полой водой, вынуждая город на все это время пристани переводить в р. Казанку, где ничтожное протяжение причальной линии, узкость и извилистость хода создают невозможные судоходные условия, которых не знает ни один город на всей Волге. Ныне Округ Путей Сообщения желает идти навстречу нуждам города и предполагает сделать спрямление реки Казанки, — конечно мера эта хорошая, но мера временная, которая между тем отнимая у города участки для спрямления реки, отнимает прекрасный выгон для пастбища скота и некоторую часть доходности берегов, которая ежегодно выражается от 1500 до 2000 рублей. Затоном же река Казанка не может служить, ибо глубокою осенью, благодаря песчаной косы и малой меженной воды, суда на зимовку в реку не будут иметь возможности взойти. Погибель же у Волжской пристани в недалеком будущем некоторых пристанских участков у всех на глазах, ибо песчаная громадная коса все ближе приближается. Чтобы дать место для пароходств, Округ должен немедленно ниже по реке Волге укрепить берег длинною по крайней мере на 300 погонных сажень, город же беря на себя заботы путем спланирования местности возвышением Волжского берега, с замощением и с земляными работами, вынужден будет потратить до 100 тысяч рублей. Чтобы получить обратно затраченный на это капитал, — нужно будет много лет, ибо удлиненная береговая линия ежегодно потребует больших расходов, как по спланировке, так равно и по ремонту мостовых по всем пути после разлива весенних вод.
	
	Возбужденный мною вопрос, как заведующего Волжскими пристанями об необходимости сооружения порта у города Казани является более, чем неотложным вопросом, в видах всего вышеизложенного, городу нельзя откладывать, иначе город с годами будет вымирать в промышленном отношении.
	
	При сооружении порта по моему проекту невозможное в настоящее время санитарное положение того большого района совсем прекратится, благодаря тому, что вся эта болотистая местность наростится высотою более 2-х сажень. О дальности Волжских пристаней, остается только одно воспоминание, ибо вместо 5 или 7 верст, река Волга от города будет отстоять только на 376 сажень, а со временем, когда нарощенная равнина этой местности застроится, то город от реки Волги будет только разделять набережная порта шириною не более ста сажень. Благодаря такому расстоянию фрахт понизится до минимума в цене, не нужно тогда думать коммерсантам, а равно и господам пароходчикам об уборке товаров с пристаней, ибо при построенных складах, во время зимы, товары будут привозиться в город по мере их надобности. Во время глубокой осени лишние рейсы будут делать пароходы, зная, что в Казанском порту имеется прекрасный затон и что потом по разгрузке товаров можно отправлять его по всем направлениям железных тупей. - Благодаря затону возникнут мастерские всевозможные и при их вырастут школы для детей всего района значит быт мастеровых улучшится в материальном положении на много, ибо он будет уже жить не на два дома, как теперь, а со своею семьей. В своей пояснительной записке не буду говорить об экономическом положении города и о его быстром росте - предоставляя судить об этом господам Гласным Городской Думы и обывателям или гражданам нашего города, которые лелеют эту мысль уже несколько десятков лет; в конце скажу только, что порт будет обслуживать кроме Казани всему нижнему плесу рек. Волге и Камскому с его немаловажными реками, — большому бассейну. По желанию некоторых господ гласных, которые уже видели мой проект приблизить его как можно более к вокзалу железной дороги, по наведенным мною справкам у одного лица, близко очень стоящего к Округу Путей Сообщения встретит затруднение, он высказался так, в виду будущего грандиозного торгового узла, Округ Путей Сообщения этого не позволит.
	
	На днях выполнив свои работы я представил 5-го февраля с.г. проект плана бухты в Городскую Управу указав, что в разработке кроме данных в мое распоряжение двух техников Управы: Н.В. Дунаева и И.Г. Саламатина, принимали отчасти участие лица близко стоящие к Округу Путей Сообщения.
	
	Согласно проекту порт представляет из себя довольно правильную фигуру длинною 750 сажень при ширине 300 сажень при низком горизонте вод.
	
	Канал, соединяющий порт с р. Волгою, разделен на 2 части: первая часть от Волги на протяжении 225 сажень, шириною в 75 сажень; вторая часть канала постепенно расширяющаяся к порту на протяжении 420 сажень, доходит до 475 саж. шириною.
	
	Правая набережная порта и канала до реки Волги стоя лицом к реке имеет в среднем 100 сажень ширины, а левая сторона доходит только до 1-ой части узкого канала при ширине 50 сажень представляя далее равнину.
	
	Порт предполагается удалить от существующих кварталов на 475 сажень, каковую местность засыпав до незаливаемости - разбить на кварталы для заселения. Береговая линия порта и канала имеет 3424 погонных сажень - из этого количества на порт приходится 1800 погонных сажень. Набережная и бермы предполагается устроить по типу Нижнего Новгорода. Высота всех насыпей, то есть кругом порта и канала и под новыми кварталами предполагается выше самого высокого уровня воды на 1\nicefrac{1}{2} аршина.
	
	Стоимость порта по подсчету выяснилось в сумме 6202075 рублей 90 коп.
	
	
	
\end{document}

