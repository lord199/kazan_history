
\documentclass[oneside,final,14pt]{extreport}
%\usepackage[koi8-r]{inputenc}
\usepackage[russianb]{babel}
\usepackage{vmargin}
\setpapersize{A4}
\usepackage[T2A]{fontenc}
\usepackage[utf8x]{inputenc}  % more recent versions (at least>=2004-17-10)
%\usepackage[russian]{babel}
\setmarginsrb{2cm}{1.5cm}{1cm}{1.5cm}{0pt}{0mm}{0pt}{13mm}
\usepackage{indentfirst}
\sloppy
\begin{document}


\section*{Историческая записка о порте у города Казани}


\begin{center}
	I
\end{center}

Вопрос о сооружении бухты в Казани был возбужден еще задолго до введения Городского Положения 16 июня 1870 года (по словам Н. Ф. Банарцева в 1858 году), но осуществление его тестно связывалось с проведением на Казань железнодорожной линии, почему и откладывалось из года в год. Тем не менее в 1872 году частной компании Арцыбушева предоставлена была возможность взяться за это предприятие. Однако предприятие это не удалось по причинам не зависившим от Городского Управления, так как на это дело потребовались средства, оказавшиеся не по силам названной компании.

После этой неудачи, постигшей частную предприимчивость, в 1886 году Городскою Думой возбуждено было ходатайство пред Правительством об устройстве бухты на средства Государственного Казначейства, но ходатайство это решено было в отрицательном смысле.

Хотя Дума снова поддерживала сношение с частными предпринимателями по сооружению бухты на концессионных началах, но все многочисленные попытки и старания к подысканию частных капиталистов не увенчались успехом.

В 1886 году вопрос о выгодности бухты для Казани в финансовом и техническом отношениях был тщательно рассмотрен и обсужден в особой коммиссии при местном Округе Путей Сообщения под председательством Начальника Округа Августовского при участии представителей: Казанской Думы, Биржевого Комитета, пароходчиков и купечества, причем Коммиссия эта пришла к единогласному решению, что бухта для Казани безусловно необходима, что она даст городу пользы до 150000 руб. в год, что в техническом отношении сооружение бухты не встречает препятствий и, что, наконец,  полная стоимость этого сооружения эллингом определится в сумму около 180000 руб. Подобная записка этой Коммиссии препровождена в Министерство Путей Сообщени, где она находится по сие время (в 1889 г. в бытность Городского Головы в Петербурге, он справлялся о судьбе этой записки - оказалось, что она еще не рассматривалась в Министерстве). 


\end{document}
