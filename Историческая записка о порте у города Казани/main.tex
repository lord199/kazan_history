
\documentclass[oneside,final,14pt]{extreport}
%\usepackage[koi8-r]{inputenc}
\usepackage[russianb]{babel}
\usepackage{vmargin}
\setpapersize{A4}
\usepackage[T2A]{fontenc}
\usepackage[utf8x]{inputenc}  % more recent versions (at least>=2004-17-10)
%\usepackage[russian]{babel}
\setmarginsrb{2cm}{1.5cm}{1cm}{1.5cm}{0pt}{0mm}{0pt}{13mm}
\usepackage{indentfirst}
\usepackage[unicode, pdftex]{hyperref}
\sloppy
\begin{document}

\section*{Описание}

{\bf Название:} Историческая записка о порте у города Казани

{\bf Издатель:} Казань Казанский федеральный университет 2013

{\bf Предметные рубрики:} Казанский речной порт -- История

{\bf Инвентарный номер}: RU05CLSL05CORRK05C1925 / c8470

{\bf Ссылка:} \hyperref[http://repo.kpfu.ru/jspui/handle/net/4754]{http://repo.kpfu.ru/jspui/handle/net/4754}


\section*{Историческая записка о порте у города Казани}


\begin{center}
	I
\end{center}

Вопрос о сооружении бухты в Казани был возбужден еще задолго до введения Городского Положения 16 июня 1870 года (по словам Н. Ф. Банарцева в 1858 году), но осуществление его тестно связывалось с проведением на Казань железнодорожной линии, почему и откладывалось из года в год. Тем не менее в 1872 году частной компании Арцыбушева предоставлена была возможность взяться за это предприятие. Однако предприятие это не удалось по причинам не зависившим от Городского Управления, так как на это дело потребовались средства, оказавшиеся не по силам названной компании.

После этой неудачи, постигшей частную предприимчивость, в 1886 году Городскою Думой возбуждено было ходатайство пред Правительством об устройстве бухты на средства Государственного Казначейства, но ходатайство это решено было в отрицательном смысле.

Хотя Дума снова поддерживала сношение с частными предпринимателями по сооружению бухты на концессионных началах, но все многочисленные попытки и старания к подысканию частных капиталистов не увенчались успехом.

В 1886 году вопрос о выгодности бухты для Казани в финансовом и техническом отношениях был тщательно рассмотрен и обсужден в особой коммиссии при местном Округе Путей Сообщения под председательством Начальника Округа Августовского при участии представителей: Казанской Думы, Биржевого Комитета, пароходчиков и купечества, причем Коммиссия эта пришла к единогласному решению, что бухта для Казани безусловно необходима, что она даст городу пользы до 150000 руб. в год, что в техническом отношении сооружение бухты не встречает препятствий и, что, наконец,  полная стоимость этого сооружения эллингом определится в сумму около 180000 руб. Подобная записка этой Коммиссии препровождена в Министерство Путей Сообщени, где она находится по сие время (в 1889 г. в бытность Городского Головы в Петербурге, он справлялся о судьбе этой записки - оказалось, что она еще не рассматривалась в Министерстве). 

В 1891 году было предложение некоего Миллиса о принятии этого сооружения к исполнениб на средства предполагавшегося к учреждению акционерного общества при условии гарантии со стороны города ежегодного валового дохода в 100000 руб. Дума, рассмотрев это предложение, признала невыгодным его для города и отклонила.

Несмотря на полное сознание всего населения г. Казани как необходимости, так и полезности соружения бухты, вопрос о ней оставался неподвижным.

Более реальную фазу вопрос о сооружении бухтиы принял в 1893 году, благодаря почину гласного Думы В. М. Ключникова. В обстроятельной записке г. Ключников, путем изучения всех имеющихся по этому вопросу данных, доказал все преимущества этого предприятия для гор. Казани, как по топографическим её условиям, так и по выгодности сооружения бухты средствами города, причем он предложил ходатайствовать, чтобы бухта была построена Министерством Путей Сообщения, но на средства города. Записка эта обсуждалась состоявшею при Городской Управе Техническою Коммиссией, при участии и содействии в разработке детальных сторон этого вопроса представителем Казанского Округа Путей Сообщения, во главе с Помощником Начальника Округа, инженером Макаровым. Подробно обсудив способы сооружения бухты, Коммиссия приняла следующие основные положения: 

\begin{enumerate} 
	\item Канал должен быть шириною в 35 саж., глубиною в 12 четвертей от низкого горизонта воды (при 0,30 ниже нуля Услонской рейки);
	\item Сама бухта должна быть удалена от границы частных владений города на 200 саж.;
	\item Ширину проездов в бухте определить во всех пунктах не менее 100 саж. низшего горизонта вод, при глубине в 12 четвертей, указанных в отношении канала в меженную воду;
	\item Длина береговой линии кругом бухты должна быть около 1500 саж.;
	\item На острове, среди бухты, должно быть достаточно места для служебных зданий, эллингов и доков;
	\item Ширина насыпи в боковых и нижних частях бухты должна быть не менее в 100 саж. и со стороны города до 200 саж.;
	\item Кругом набережная должна быть устроена по типу Нижегородской набережной с двойным объездом, при этом, однако, высоту нижней бермы следует взять соответственно естественному возвышению местностим с тем, чтобы впоследствии, в случае надобности, она могла быть понижена. С нижней бермы должно быть сделано по крайней мере два пологих двойных въезда на верхнюю набережную. Ширина бермы должна быть в 10 саж.;
	\item  Ширина мостовых определяет так: а) весь нижний откос от воды, б) 6 сажень нижней бермы под дорогу и в) верхний путь не менее 10 сажень;
	\item Признавая всю важность и полезность эллингов и доков, в виду их дороговизны, предоставить устройство их, одновременно с сооружением бухты, частной предприимчивости: вызвав на эти сооружениыя особых предпринимателей;
	\item Высота всех насыпей, кругом бухты должна быть выше высокого уровня воды, бывшего в 1867 году, на 6 четвертей;
	\item От города должны быть сделаны следубщие сооружения: дом для администрации, будки, фонари, пожарное депо с каланчей и другие необходимые сооружения.
\end{enumerate}

Обращаясь к материальной стороне вопроса, Коммиссия пришла к решительному и единогласному заключению, что единственный, более рациональный путь в данном случае для интересов и пользы дела {\it строить бухту на счет города, средствами займа}, причем путем подробных исчислений особой подкоммиссии; дозодность бухты установлена следующими цифрами: с земель набережной 90000 руб., попудный сбор 50000 руб. и за зимовку судов 8000 руб., а всего 148000 руб. Расходы же по эксплуатации бухты ожидаются по вычислению подкоммиссии следующие: а) процентов на капитал (принимая стоимость сооружения в 100000 руб.) с погашением около 50000 руб., собственно расходов на эксплуатации вместе с ремонтов 35000 руб., всего 85000 руб. Таким образом, если исключить из ожидаемой прибыли от бухты доходы, получаемые с пристаней (в 1893 г., во время составления настоящего рассчета), даже при крайне осторожном исчислении, как доходов, так равно и расходов, - получится полнейшая выгода для города от означенного предприятия.

Руководствуясь принятыми выше соображениями, Коммиссия, в виду крайне сочувственного отношения местного Округа Путей Сообщения к делу осуществления означенного предприятия, обратилась к г. Начальнику Округа с просьбою поручить инженерам вверенного ему Округа составить проект сооружения бухты на основании данных, выработанных Коммиссией с помощью принимавших участие в её трудах специалистов.

Начальник Округа Лохтин, препровождая разработанный проект бухты, составленный инженерами Округа, в письме на имя бывшего в то время Городским Главой С. В. Дьяченко, выразио удовольствие, что в данном случае ему удалось быть полезным в деле осуществления этого предприятия, которое он надеется "при нынешнем" сочувствии этому делу со стороны Городской Думы и сознания всего его значения как для города. так и для судоходства, не встретив в дальнейшем каких-либо препятствий или отсрочек. Не подлежит сомнению, писал г. Лохтин, что порт этот стоящий на рубеже двух волжских плес, верхнего и нижнего, естественным образом различающихся своим характером и глубиновую и примыкающий с одной стороны к вновь открытой железной дороге, которая соединяет его непосредственно с Москвою и с другой - с восточными рынками Камского бассейна, послужив весьма важным сооружением для всего волжского судоходого движения и, в то же время, щедро вознаградить за заботы о своем осуществлении и сам город, оживлением его торговли, открытием механических мастерских и заводов для ремонта или постройки судов, привлечением к порту многочисленного нового народонаселения и т.п. Не даром вопрос о порте в Казани, возникший почти полстоления тому назад, благодаря своей жизненности, не переставал время от времени обращать на себя внимание и создал за этот период времени целый ряд проектов и предложения по поводу осуществления этого дела. Остается выразить только удивление, дальше пишет г. Лохтин, что столь серьезный и так сказать назревший городской вопрос оставался до сего времени не разрешенным, не смотря на очевидую необходимость в этом сооружении с точки зрения городских интересов. Препровождаемый проект, составлен на основании тщательных и подробных исследований местности и грунтов и разработан при участии большинства Окружных Инженеров, обсуждавших его в общих своих совещаниях, а потому г. Лохтин рекомендует этот проект, как достаточно гарантирующий правильность его разработки с технической точки зрения. Общая стоимость проекта высчитана в сумме 1157142 руб.


\end{document}
